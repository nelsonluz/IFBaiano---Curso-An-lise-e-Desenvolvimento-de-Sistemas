% Instituto Federal de Educação, Ciência e Tecnologia Baiano - Campus Guanambi
% 
% Modelo para Trabalho de Conclusão de Curso em LaTeX
% Superior de Análise e Desenvolvimento de Sistemas
% Alterado por: Dr. Naidson Clayr Santos Ferreira
%
% ----------------------------------------------------------------------- %
% Arquivo: pacotes.tex
% ----------------------------------------------------------------------- %

% REFERÊNCIAS------------------------------------------------------------------
\usepackage[%
    alf,
    abnt-emphasize=bf,
    bibjustif,
    recuo=0cm,
    abnt-url-package=url,       % Utiliza o pacote url
    abnt-refinfo=yes,           % Utiliza o estilo bibliográfico abnt-refinfo
    abnt-etal-cite=3,
    abnt-etal-list=3,
    abnt-thesis-year=final
]{abntex2cite}                  % Configura as citações bibliográficas conforme a norma ABNT

% PACOTES----------------------------------------------------------------------
\usepackage{amssymb}
% símbolos matemáticos

% Codificação do documento
\usepackage[utf8]{inputenc}
     
% Seleção de código de fonte                           
\usepackage[T1]{fontenc}
                                    
% Réguas horizontais em tabelas
\usepackage{booktabs}

% Controle das cores                                       
\usepackage{color, colortbl}   

% Necessário para tabelas/figuras em ambiente multi-colunas
\usepackage{float}

% Inclusão de gráficos e figuras
\usepackage{graphicx}

% Uso de vírgulas em expressões matemáticas
\usepackage{icomma}

% Indenta o primeiro parágrafo de cada seção
\usepackage{indentfirst}

% Melhora a justificação do documento
\usepackage{microtype}

% Permite tabelas com múltiplas linhas e colunas
\usepackage{multirow, array}

% Permite subnumeração de equações
\usepackage{subeqnarray}

% Para encontrar última página do documento                                    
\usepackage{lastpage} 

% Permite apresentar texto tal como escrito no documento, ainda que sejam comandos Latex                                      
\usepackage{verbatim}

% Fontes e símbolos matemáticos                                       
\usepackage{amsfonts, amssymb, amsmath}
             
% Permite escrever algoritmos em português                     
\usepackage[algoruled, portuguese]{algorithm2e}
             
% Usa a fonte Helvetica             
%\usepackage[scaled]{helvet}
 
% Usa a fonte Times                                
\usepackage{times}
                                         
% Usa a fonte Palatino
%\usepackage{palatino}

% Usa a fonte Latin Modern                                      
%\usepackage{lmodern}

% Mantém as notas de rodapé sempre na mesma posição                                       
\usepackage[bottom]{footmisc}
        
% Fontes de alta qualidade                              
\usepackage{ae, aecompl}

% Símbolos matemáticos                                    
\usepackage{latexsym}

% Permite páginas em modo "paisagem"                                       
\usepackage{lscape}
 
% Dispor imagens em parágrafos                                         
%\usepackage{picinpar}                                      

%Tabela Colorida
\usepackage{colortbl}
\usepackage{xcolor}

%Tabelas Longas
\usepackage{longtable}
\usepackage[graphicx]{realboxes}
%\usepackage{scalefnt}                                      % Permite redimensionar tamanho da fonte
%\usepackage{subfig}                                        % Posicionamento de figuras
%\usepackage{upgreek}                                       % Fonte letras gregas

% Redefine a fonte para uma fonte similar a Arial (fonte Helvetica)
%\renewcommand*\familydefault{\sfdefault}
